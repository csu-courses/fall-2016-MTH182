\documentclass[handout]{ximera}

%\usepackage{todonotes}

\newcommand{\todo}{}


\graphicspath{
{./}
{../functionsOfSeveralVariables/}
}


\usepackage{tkz-euclide}
\tikzset{>=stealth} %% cool arrow head
\tikzset{shorten <>/.style={ shorten >=#1, shorten <=#1 } } %% allows shorter vectors

\usetikzlibrary{backgrounds} %% for boxes around graphs
\usetikzlibrary{shapes,positioning}  %% Clouds and stars
\usetikzlibrary{matrix} %% for matrix
\usepgfplotslibrary{polar} %% for polar plots
\usetkzobj{all}
\usepackage[makeroom]{cancel} %% for strike outs
%\usepackage{mathtools} %% for pretty underbrace % Breaks Ximera
\usepackage{multicol}





\usepackage{array}
\setlength{\extrarowheight}{+.1cm}   
\newdimen\digitwidth
\settowidth\digitwidth{9}
\def\divrule#1#2{
\noalign{\moveright#1\digitwidth
\vbox{\hrule width#2\digitwidth}}}





\newcommand{\RR}{\mathbb R}
\newcommand{\R}{\mathbb R}
\newcommand{\N}{\mathbb N}
\newcommand{\Z}{\mathbb Z}

%\renewcommand{\d}{\,d\!}
\renewcommand{\d}{\mathop{}\!d}
\newcommand{\dd}[2][]{\frac{\d #1}{\d #2}}
\newcommand{\pp}[2][]{\frac{\partial #1}{\partial #2}}
\renewcommand{\l}{\ell}
\newcommand{\ddx}{\frac{d}{\d x}}

\newcommand{\zeroOverZero}{\ensuremath{\boldsymbol{\tfrac{0}{0}}}}
\newcommand{\inftyOverInfty}{\ensuremath{\boldsymbol{\tfrac{\infty}{\infty}}}}
\newcommand{\zeroOverInfty}{\ensuremath{\boldsymbol{\tfrac{0}{\infty}}}}
\newcommand{\zeroTimesInfty}{\ensuremath{\small\boldsymbol{0\cdot \infty}}}
\newcommand{\inftyMinusInfty}{\ensuremath{\small\boldsymbol{\infty - \infty}}}
\newcommand{\oneToInfty}{\ensuremath{\boldsymbol{1^\infty}}}
\newcommand{\zeroToZero}{\ensuremath{\boldsymbol{0^0}}}
\newcommand{\inftyToZero}{\ensuremath{\boldsymbol{\infty^0}}}



\newcommand{\numOverZero}{\ensuremath{\boldsymbol{\tfrac{\#}{0}}}}
\newcommand{\dfn}{\textbf}
%\newcommand{\unit}{\,\mathrm}
\newcommand{\unit}{\mathop{}\!\mathrm}
\newcommand{\eval}[1]{\bigg[ #1 \bigg]}
\newcommand{\seq}[1]{\left( #1 \right)}
\renewcommand{\epsilon}{\varepsilon}
\renewcommand{\iff}{\Leftrightarrow}

\DeclareMathOperator{\arccot}{arccot}
\DeclareMathOperator{\arcsec}{arcsec}
\DeclareMathOperator{\arccsc}{arccsc}
\DeclareMathOperator{\si}{Si}
\DeclareMathOperator{\proj}{proj}
\DeclareMathOperator{\scal}{scal}


\newcommand{\tightoverset}[2]{% for arrow vec
  \mathop{#2}\limits^{\vbox to -.5ex{\kern-0.75ex\hbox{$#1$}\vss}}}
\newcommand{\arrowvec}[1]{\tightoverset{\scriptstyle\rightharpoonup}{#1}}
\renewcommand{\vec}{\mathbf}
\newcommand{\veci}{\vec{i}}
\newcommand{\vecj}{\vec{j}}
\newcommand{\veck}{\vec{k}}
\newcommand{\vecl}{\boldsymbol{\l}}

\newcommand{\dotp}{\bullet}
\newcommand{\cross}{\boldsymbol\times}
\newcommand{\grad}{\boldsymbol\nabla}
\newcommand{\divergence}{\grad\dotp}
\newcommand{\curl}{\grad\cross}
%\DeclareMathOperator{\divergence}{divergence}
%\DeclareMathOperator{\curl}[1]{\grad\cross #1}
\newcommand{\lto}{\mathop{\longrightarrow\,}\limits}


\colorlet{textColor}{black} 
\colorlet{background}{white}
\colorlet{penColor}{blue!50!black} % Color of a curve in a plot
\colorlet{penColor2}{red!50!black}% Color of a curve in a plot
\colorlet{penColor3}{red!50!blue} % Color of a curve in a plot
\colorlet{penColor4}{green!50!black} % Color of a curve in a plot
\colorlet{penColor5}{orange!80!black} % Color of a curve in a plot
\colorlet{fill1}{penColor!20} % Color of fill in a plot
\colorlet{fill2}{penColor2!20} % Color of fill in a plot
\colorlet{fillp}{fill1} % Color of positive area
\colorlet{filln}{penColor2!20} % Color of negative area
\colorlet{fill3}{penColor3!20} % Fill
\colorlet{fill4}{penColor4!20} % Fill
\colorlet{fill5}{penColor5!20} % Fill
\colorlet{gridColor}{gray!50} % Color of grid in a plot

\newcommand{\surfaceColor}{violet}
\newcommand{\surfaceColorTwo}{redyellow}
\newcommand{\sliceColor}{greenyellow}




\pgfmathdeclarefunction{gauss}{2}{% gives gaussian
  \pgfmathparse{1/(#2*sqrt(2*pi))*exp(-((x-#1)^2)/(2*#2^2))}%
}


%%%%%%%%%%%%%
%% Vectors
%%%%%%%%%%%%%

%% Simple horiz vectors
\renewcommand{\vector}[1]{\left\langle #1\right\rangle}


%% %% Complex Horiz Vectors with angle brackets
%% \makeatletter
%% \renewcommand{\vector}[2][ , ]{\left\langle%
%%   \def\nextitem{\def\nextitem{#1}}%
%%   \@for \el:=#2\do{\nextitem\el}\right\rangle%
%% }
%% \makeatother

%% %% Vertical Vectors
%% \def\vector#1{\begin{bmatrix}\vecListA#1,,\end{bmatrix}}
%% \def\vecListA#1,{\if,#1,\else #1\cr \expandafter \vecListA \fi}

%%%%%%%%%%%%%
%% End of vectors
%%%%%%%%%%%%%

%\newcommand{\fullwidth}{}
%\newcommand{\normalwidth}{}



%% makes a snazzy t-chart for evaluating functions
%\newenvironment{tchart}{\rowcolors{2}{}{background!90!textColor}\array}{\endarray}

%%This is to help with formatting on future title pages.
\newenvironment{sectionOutcomes}{}{} 

\title{Accumulated cross-sections}


\begin{document}
\begin{abstract}
  We can also use integrals to compute volume and mass.
\end{abstract}
\maketitle

\section{Volume}

We have seen how to compute certain areas by using integration. Now we
will see how to compute some volumes.  The volumes that we can compute
will have cross-sections that are easy to describe. Sometimes we think
of these cross-sections as being ``slabs'' that we are layering to
create a volume.


\begin{question}
Find the volume of a pyramid with a square base that is $20$ meters tall
and $20$ meters on a side at the base.
\begin{image}
  \begin{tikzpicture}
    \begin{axis}[
          xmin =-4,xmax=7,ymax=4,ymin=-4,
          axis lines=center, xlabel=$x$, ylabel=$y$,
          every axis y label/.style={at=(current axis.above origin),anchor=south},
          every axis x label/.style={at=(current axis.right of origin),anchor=west},
          width=5in,
          axis on top,
          xtick={0,6}, xticklabels={$0$, $20$},
          ytick={0,3},yticklabels={$0$,$20$},
            clip=false,
      ]
      \addplot [draw=penColor, fill = fill1, very thick] plot coordinates {(-3,-3) (3,-3) (6,0) (1.5, 3) (-3,-3)};
      \addplot [draw=penColor, very thick] plot coordinates {(-3,-3) (0,0)};
      \addplot [draw=penColor, very thick] plot coordinates {(6,0) (0,0)};
      \addplot [draw=penColor, very thick] plot coordinates {(1.5,3) (3,-3)};
      \addplot [draw=penColor, very thick] plot coordinates {(1.5,3) (0,0)};

      \addplot [->] plot coordinates {(0,0) (-4,-4)};
      \node[anchor=north east] at (axis cs:-4,-4) {$z$};       
    \end{axis}
  \end{tikzpicture}
\end{image}



\begin{explanation}
To solve this problem, we will ``sum up'' (integrate) infinitely many
``infinitesimal'' slabs which are parallel to the base of the pyramid
to obtain the volume.
\begin{image}
  \begin{tikzpicture}
    \begin{axis}[
          xmin =-4,xmax=7,ymax=4,ymin=-4,
          axis lines=center, xlabel=$x$, ylabel=$y$,
          every axis y label/.style={at=(current axis.above origin),anchor=south},
          every axis x label/.style={at=(current axis.right of origin),anchor=west},
          axis on top,
          width=5in,
          xtick={0,6}, xticklabels={$0$, $20$},
          ytick={0,3},yticklabels={$0$,$20$},
            clip=false,
      ]
      \addplot [draw=penColor, thick] plot coordinates {(-3,-3) (3,-3) (6,0) (1.5, 3) (-3,-3)};
            
      \addplot [draw=penColor, thick] plot coordinates {(-3,-3) (0,0)};
      \addplot [draw=penColor, thick] plot coordinates {(6,0) (0,0)};
      \addplot [draw=penColor, thick] plot coordinates {(1.5,3) (3,-3)};
      \addplot [draw=penColor, thick] plot coordinates {(1.5,3) (0,0)};

      %% slab
      \addplot [draw=penColor, fill=fill1,very thick] plot coordinates {(3,2) (1,2) (0,1) (2, 1) (3,2)};
      \addplot [draw=penColor, fill=fill1,very thick] plot coordinates {(0,.8) (0,1) (2,1) (2, .8) (0,.8)};
      \addplot [draw=penColor, fill=fill1,very thick] plot coordinates {(2,1) (2, .8) (3,1.8) (3,2) (2,1)};

      %\addplot [draw=penColor, fill=fill1,very thick] plot coordinates {(3,1.8) (1,1.8) (0,.8) (2, .8) (3,1.8)};
      %\addplot [draw=penColor, fill=fill1,very thick] plot coordinates {(3,2) (1,2) (0,1) (2, 1) (3,2)};

      \addplot [draw=penColor, thick] plot coordinates {(1.5,3) (3,-3)};

      \draw[decoration={brace,mirror,raise=.1cm},decorate,thin] (axis cs:0,.8)--(axis cs:2,.8);
      \draw[decoration={brace,raise=.1cm},decorate,thin] (axis cs:3.1,2.05)--(axis cs:3.1,1.75);
      
      \addplot [->] plot coordinates {(0,0) (-4,-4)};
      \node[anchor=north east] at (axis cs:-4,-4) {$z$};

      \node at (axis cs:3.6,1.9) {$\d y$};
      \node at (axis cs:1,.4) {$L(y)$};       
    \end{axis}
  \end{tikzpicture}
\end{image}

The ``height'' of each slab will be $\d y$, 

Let  $L(y)$ be the width of the slab at height $y$, then $ L(y) =  $

\vspace{0.5 in}

So the volume of the infinitesimal slab at height $y$ is:  

\vspace{0.5 in}

So the total volume is given by the integral:

\vspace{0.5 in}

We can evaluate this integral:

\vspace{1 in}

\end{explanation}

\end{question}

\newpage



\begin{question}
The base of a solid is the region bounded by $f(x)=x^2-1$ and
$g(x)=-x^2+1$:
\begin{image}
\begin{tikzpicture}
  \begin{axis}[
      xmin=-1, xmax=1,domain=-1:1,
      clip=false,
      axis lines =center, xlabel=$x$, ylabel=$y$,
      every axis y label/.style={at=(current axis.above origin),anchor=south},
      every axis x label/.style={at=(current axis.right of origin),anchor=west},
      axis on top,
    ] 
    \addplot [penColor2,very thick] {x^2-1};
    \addplot [penColor,very thick] {-x^2+1};
    \node at (axis cs:1,0.8) [penColor] {$y = -x^2+1$};
    \node at (axis cs:1,-0.8) [penColor2] {$y = x^2-1$};
  \end{axis}
\end{tikzpicture}
\end{image}
The cross-section of this solid consists of equilateral triangles
that are perpendicular to the $x$-axis:
\begin{image}
\begin{tikzpicture}
  \begin{axis}[
      xmin=-1.2, xmax=1.2,domain=-1:1,
      ymin=-.4,ymax=1,
      clip=false,
      ytick={0,.8},yticklabels={$0$,$1$},
      axis lines =center, xlabel=$x$, ylabel=$z$,
      every axis y label/.style={at=(current axis.above origin),anchor=south},
      every axis x label/.style={at=(current axis.right of origin),anchor=west},
      axis on top,
    ] 
    \addplot [penColor2,very thick] {(0.3)*x^2-(0.3)};
    \addplot [penColor,very thick] {(-0.3)*x^2+(0.3)};

    \addplot [draw=penColor, fill=fill1,very thick] plot coordinates {(-.11,-.3) (.1,.3) (0,.8) (-.11,-.3)};

    \addplot [draw=penColor, fill=fill1,very thick] plot coordinates {(-.67,-.17) (-.53,.21) (-.6,.6) (-.67,-.17)};

    \addplot [draw=penColor, fill=fill1,very thick] plot coordinates {(.53,-.22) (.67,.16) (.6,.6) (.53,-.22)};


    \addplot [black,very thick] {-.28 + .0086*sqrt(16031-14948*x^2)};

    \addplot [->] plot coordinates {(-.6,-.8) (.6,0.8)};
    \node[anchor=south west] at (axis cs:.6,0.8) {$y$};
  \end{axis}
\end{tikzpicture}
\end{image}
Find the volume of this solid.
\begin{explanation}
We want to find the volume of the triangular slab at a given $x$
value. We know the width of each slab is $\d x$:
\begin{image}
\begin{tikzpicture}
  \begin{axis}[
      xmin=-1.2, xmax=1.2,domain=-1:1,
      ymin=-.4,ymax=1,
      clip=false,
      ytick={0,.8},yticklabels={$0$,$1$},
      axis lines =center, xlabel=$x$, ylabel=$z$,
      every axis y label/.style={at=(current axis.above origin),anchor=south},
      every axis x label/.style={at=(current axis.right of origin),anchor=west},
      axis on top,
    ] 
    \addplot [penColor2,very thick] {(0.3)*x^2-(0.3)};
    \addplot [penColor,very thick] {(-0.3)*x^2+(0.3)};

    \addplot [draw=penColor, fill=fill1,very thick] plot coordinates {(-.11,-.3) (.1,.3) (0,.8) (-.11,-.3)};

    \addplot [draw=penColor, fill=fill1,very thick] plot coordinates {(-.11,-.3) (-.19,-.3) (-.08,.8) (0,.8) (-.11,-.3)};

    \draw[decoration={brace,raise=.1cm},decorate,thin] (axis cs:-.11,-.3)--(axis cs:-.19,-.3);  
    \node[anchor=north] at (axis cs:.-.15,-.35) {$\d x$};
    
    \addplot [black,very thick] {-.28 + .0086*sqrt(16031-14948*x^2)};

    \addplot [->] plot coordinates {(-.6,-.8) (.6,0.8)};
    \node[anchor=south west] at (axis cs:.6,0.8) {$y$};
  \end{axis}
\end{tikzpicture}
\end{image}

We have a different ``infinitesimal slab'' for each $x$, each with width $\d x$.

At $x$, the length of one of the sides of the slabs is:

\vspace{0.5 in}

Using geometry, find the area of the face of the slab at $x$:

\vspace{1 in}

So the volume of the infinitesimal slab at $x$ is:

\vspace{0.5 in}

Thus the volume of the solid is given by the integral;

\vspace{0.5 in}
\newpage
We can evaluate this integral:

\vspace{7 in}


\end{explanation}
\end{question}

\section{Density and Mass}

Given an object which can be modeled as one-dimensional (like a wire),
its \dfn{linear density} is a measure of its mass per unit length.

\begin{question}
	If a wire has linear density $5 \unit{g}/\unit{m}$, how many grams
	will $4 \unit{m}$ of this wire be?
	
	\[
	\mathrm{Mass} = 
	\]
	
\end{question}

Sometimes the linear density of an object can vary from one part of
the object to another. In this case density will be a function, that
is usually represented by the Greek letter, \textit{rho}, $\rho$:
\[
\rho(x) = \text{density with respect to position.}
\]

In this case, we can think of breaking the object up into infinitesminal sections of length $dx$. 
The mass of each segment will be $\rho(x) \d x$, and we want to sum these from the initial point $x=a$ to $x=b$.  

Thus

\[
\mathrm{Mass} = \int_{a}^{b} \rho(x) \d x
\]
where $a$ is the starting position and $b$ is the ending position.

\begin{question}
	If the linear density of a rod of length $5\unit{m}$ is given by
	$\rho(x) = 1+3x^2 \unit{g}/\unit{m}$, where $x$ is the distance from
	one end of the rod, then what is total mass of the rod?
	\begin{prompt}
		\[
		\mathrm{Mass} = \answer{130} \unit{g}
		\]
	\end{prompt}
\end{question}


\end{document}