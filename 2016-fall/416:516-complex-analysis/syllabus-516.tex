%  Initial document from Ivan Supranov 

\documentclass[11pt]{article}
\setlength{\topmargin}{-.7in}
\setlength{\oddsidemargin}{0in}
\setlength{\textwidth}{16cm}
\setlength{\textheight}{23cm}

\pagestyle{empty}

\begin{document}
	\begin{center}
		{\noindent \bf \large MTH~416 Complex Analysis\\
			Fall 2012 -- Course Information\\
		}
	\end{center}
	
	%\medskip
	\medskip
	
	\begin{tabbing}
		{\bf Instructor:} Steven Gubkin \hspace{1.5in} \=
		{\bf Office:} Rhodes Tower 1555\\ 
		{\bf E-mail:} {\tt s.gubkin@csuohio.edu} \>
		{\bf Office phone:} 216-687-4707\\ 
		{\bf Course Web page:}  via Blackboard Learn  \> 
		{\bf Office hours:} Tu 2:30--3:30 PM,\\ 
		
		\noindent {\bf Lectures:} TuTh 4:00--5:50 PM in MC 316\\
		
		\noindent {\bf Textbook:} Marsden and Hoffman, {\it
			Basic Complex Analysis}, 3rd edition. \\  
	\end{tabbing}

%\noindent {\bf Course Description:} %From catalogue
%Topics to be covered include applications of integration, techniques of integration, improper integrals, infinite series, power series, polar coordinates,  and vectors.\\

%\noindent {\bf Prerequisite:} A grade of `C' or better in MTH 281 or MTH 283, a grade of `C' or better in at least one mathematics course numbered 300 or above, or permission of the instructor.\\

%\noindent {\bf Schedule:} 
%A tentative course schedule will be handed out. It is subject to
%revision at any time. We will cover Chapters
%1-4. \\
 
\noindent {\bf Course Description:}  This course deals with the fundamentals of complex analysis, including basic properties of complex numbers, analytic functions, harmonic functions, integration, Taylor and Laurent series, residue calculus and conformal mapping, and their applications.\\
 
\noindent {\bf About the Course:} This is an advanced math course which will require
deep understanding of new definitions and concepts as well as mastering new techniques. 
When you learn them, pay particular attention to {\it concrete examples}.
Memorizing definitions without being able to give examples simply won't work.
You will see both computations and proofs in class and in your homework.
This will help you learn to think and express yourselves mathematically and 
to improve your mathematical writing and oral presentation skills.\\

\noindent {\bf Attendance/Class participation:}
I expect that  you attend every class and take notes. Repeated
unexcused absences will lower your participation grade.
Also active participation is encouraged and will contribute
positively to your grade. Such participation includes answering
questions in class, asking good questions in class, presenting
homework solutions, etc. \\

\noindent {\bf Exams:} 
There will be two in-class exams and a comprehensive final exam (see
syllabus for dates). \\

\noindent {\bf Make-Up Policy:} 
If you are unable to take an exam at the scheduled time due to
sickness or an emergency, notify me as soon as possible and we will
work out an appropriate accommodation. If you have a non-emergency
conflict, you must notify me at least a week in
advance in order to have the possibility of making other arrangements.\\

\noindent {\bf Homework:} 
Doing the homework is essential to understanding the material. 
Work on the material outside class. Carefully read the textbook and lecture notes {\it before}
coming to class. While reading use pencil and paper to work on examples and fill in
omitted steps in proofs.  You should seek help when you have trouble.

Solving the assigned problems and writing up the solutions will require a fair amount of time.
I encourage you to work with other students from the class, but each student
must write their solutions independently! 

% Your solutions
%should be written in full grammatically correct sentences. 
 When doing calculations, the goal is to explain your
method, not just obtain an answer. When writing a proof, you need to put in enough detail
for the reader to follow your logic easily.
Write legibly. If you are typing, you should use a 12 point
font with 1 inch margins. Staple the pages (in proper order) together.

I will collect homework every Tuesday (starting Sep 6) at the beginning of class. 
No late homework will be accepted!\\


%\newpage

 
\noindent {\bf Blackboard:} We are using Blackboard Learn in this course.
You must log in regularly. There are discussion areas,
grades, handouts, etc. You were automatically signed up when
you registered for this course.\\ To access Blackboard go to
{\tt http://csuohio.edu/elearning/blackboard/bbindex.html} and choose Blackboard Learn
or go directly to
{\tt https://bblearn.csuohio.edu/MACAuth/login.jsp}. Log in using your
CampusPass. See me if you have trouble.\\


\noindent {\bf Grades:} 
Your grade will be determined as follows. 

\medskip


\begin{tabular}{rlp{4cm}ll}
	& & & A & 90\% -- 100\%\\
	5\% & Class participation & & A$-$ & 85\% -- 89\%\\
	25\% & Homework & & B$+$ & 80\% -- 84\%\\
	20\% &  Exam I & & B & 77\% -- 79\%\\ 
	20\% & Exam II & & B$-$ & 74\% -- 76\%\\
	30\% & Final Exam  & & C$+$ & 70\% -- 73\%\\
	------ & ------------------------ & & C & 60\% -- 69\%\\
	100\% & TOTAL  & & D & 50\% -- 59\%\\ 
	& & & F & below 50\%
\end{tabular}

\bigskip


\noindent {\bf Getting help:}
There are several options for getting help with this course:

$\bullet$ Ask questions about the material or lecture during
class. Chances are that other students are also confused and a brief
question in class may save you hours of misery later.
  
$\bullet$ Ask questions of your peers, either in person or using the
Discussions area in Blackboard. 

$\bullet$ Come to my office during office hours. You should be
prepared with specific questions. Try to do as much as possible to
pinpoint your confusion before coming to see me so that we can make
effective use of our time together. Bring any work you have done so
far. If you cannot attend my office hours, feel free to contact me and
make an appointment.
  
$\bullet$ For general academic help the Focus Center (UC 563,
687-5114) runs workshops about time management, test taking skills,
study skills, etc. \\

\noindent{\bf Accommodations for students with disabilities:} CSU provides 
classroom accommodations, auxiliary aids and services to ensure equal educational
opportunities for all students regardless of their disability. For more information contact
the Office of Disability Services at 216-687-2015. Accommodations need to be
requested in advance and will not be granted retroactively.\\

\noindent {\bf Scholastic Dishonesty:} Cheating of any form is not
acceptable and it will be dealt with harshly if detected. 
Copying work done by others, in or out of class, is an act of
scholastic dishonesty and it will be prosecuted to the full extent
allowed by university policy. 
CSU policy regarding cheating and plagiarism can be found in the CSU
Code of Student Conduct at
{\tt http://www.csuohio.edu/studentlife/StudentCodeOfConduct2004.pdf}\\

\noindent {\bf Differences between MTH 416 and MTH 516:}
MTH 416 will end on Nov 10th, while MTH 516 will continue until the end of the semester.
Graduate students will be required to do more work and will be held a higher
qualitative standard.  They should synthesize and apply knowledge at a
more sophisticated level.  They should
demonstrate in writing a more sophisticated understanding of the
course material.
This is reflected in the course by: (a) homework will have more challenging problems for
graduate students and (b) tests for graduate students will be distinct from tests for
undergraduate students.

\newpage
\setlength{\topmargin}{0in}
\setlength{\oddsidemargin}{0in}
\setlength{\textwidth}{6.5in}
\setlength{\textheight}{9.2in}
\setlength{\parindent}{0in}
\pagestyle{empty}

\begin{center}
\begin{bf}
\noindent
{\large Math~516 Complex Analysis -- Fall 2016 -- Weekly Syllabus}\\
\end{bf}
\end{center}

This is a tentative syllabus as of 08/27/16. Lecture topics are subject
to change without notice. Exam days and topics 
are also subject to change, but any changes will be
announced at least a week in advance.\\
  
{\bf Text:} Marsden and Hoffman, {\it
  Basic Complex Analysis}, 3rd edition.\\
(For errata see \verb|http://www.cds.caltech.edu/~marsden/books/Basic_Complex_Analysis.html|) 
 
\begin{tabbing}
{\bf Exam Dates: }
\= Thursday, October 6, in class \hspace{1in} \= Exam 1\\ 
\> Tuesday, November 10, in class  \> Exam 2\\
\> Tuesday, December 13, 4--6 PM \> Final Exam (cumulative)\\
\end{tabbing}


{\bf Week 1: Aug 30, Sep 1}\\
1.1 Introduction to Complex Numbers\\
1.2 Properties of Complex Numbers\\
H Geometry of Complex Functions\\


{\bf Week 2: Sep 6, 8}\\ 
1.3 Some Elementary Functions\\
1.4 Continuous Functions\\
\textit{September 9th is the last day to drop}\\

{\bf Week 3: Sep 13, 15}\\ 
H Review of the Derivative in Several Real Dimensions, Wirtinger Derivatives\\
1.5 Basic Properties of Analytic Functions\\
1.6 Differentiation of the Elementary Functions\\

{\bf Week 4: Sep. 20--22}\\ 
H Review of Line Integrals, Double Integrals, and Greens Theorem\\
2.1 Contour Integrals\\
2.2 Cauchy's Theorem-A First Look\\

{\bf Week 5: Sep. 27--29}\\
2.3 A Closer Look at Cauchy's Theorem\\
2.4 Cauchy's Integral Formula\\

{\bf Week 6: Oct. 4--6}\\
Q\&A for exam\\
EXAM I on Thursday\\ 
\newpage

{\bf Week 7: Oct. 11--13}\\
\textit{Tuesday, Oct 11th, is Veterans day; no classes}
2.5 Maximum Modulus Theorem and Harmonic Functions\\
H Local Structure of Complex Functions and the Open Mapping Theorem\\

{\bf Week 8: Oct. 18--20}\\
3.1 Convergent Series of Analytic Functions\\
3.2 Power Series and Taylor's Theorem\\

{\bf Week 9: Oct. 25--27}\\
3.3 Laurent Series and Classification of Singularities  \\

{\bf Week 10: Nov. 1--3}\\
4.1 Calculation of Residues\\
4.2 Residue Theorem\\
4.3 Evaluation of Definite Integrals\\
{\it  Friday Nov 4th is the last day to withdraw}\\

{\bf Week 11: Nov. 8--10}\\
Q\&A for exam\\ 
EXAM II on Thursday\\ 
\textit{Nov 10th is the last day of class for 416 students}\\

\begin{center}
	\line(1,0){450}
\end{center}


{\bf Week 12: Nov. 15--17}\\
6.2 Rouche's Theorem and Principle of the Argument\\
6.3 Mapping Properties of Analytic Functions\\

{\bf Week 13: Nov. 22--24}\\
8.1 Basic Properties of Laplace Transforms\\
8.2 Complex Inversion Formula\\
8.3 Applications of Laplace Transforms to Ordinary Differential Equations\\

{\bf Week 14: Nov. 29--Dec 1}\\
H Automorphisms of the Plane, Riemann Sphere and Disk\\
H The Poincare Disk\\

{\bf Week 15: Dec 6--Dec 8}\\
H The Riemann Mapping theorem\\
Review\\

{\bf Final Exam: Tuesday, Dec 13, 4--6 PM}

\end{document}


