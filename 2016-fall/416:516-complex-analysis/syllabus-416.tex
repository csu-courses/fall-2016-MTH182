
%  Initial document from Ivan Supranov 

\documentclass[11pt]{article}
\setlength{\topmargin}{-.7in}
\setlength{\oddsidemargin}{0in}
\setlength{\textwidth}{16cm}
\setlength{\textheight}{23cm}

\pagestyle{empty}

\begin{document}
\begin{center}
{\noindent \bf \large MTH~416 Complex Analysis\\
 Fall 2012 -- Course Information\\
}
\end{center}

%\medskip
\medskip

\begin{tabbing}
{\bf Instructor:} Steven Gubkin \hspace{1.5in} \=
{\bf Office:} Rhodes Tower 1555\\ 
{\bf E-mail:} {\tt s.gubkin@csuohio.edu} \>
{\bf Office phone:} 216-687-4707\\ 
{\bf Course Web page:}  via Blackboard Learn  \> 
{\bf Office hours:} Tu 2:30--3:30 PM,\\ 

\noindent {\bf Lectures:} TuTh 4:00--5:50 PM in MC 316\\

\noindent {\bf Textbook:} Marsden and Hoffman, {\it
	Basic Complex Analysis}, 3rd edition. \\  
\end{tabbing}


%\noindent {\bf Course Description:} %From catalogue
%Topics to be covered include applications of integration, techniques of integration, improper integrals, infinite series, power series, polar coordinates,  and vectors.\\

\noindent {\bf Prerequisite:} A grade of `C' or better in MTH 281 or MTH 283, a grade of `C' or better in at least one mathematics course numbered 300 or above, or permission of the instructor.\\

%\noindent {\bf Schedule:} 
%A tentative course schedule will be handed out. It is subject to
%revision at any time. We will cover Chapters
%1-4. \\
 
\noindent {\bf Course Description:}  This course deals with the fundamentals of complex analysis, including basic properties of complex numbers, analytic functions, harmonic functions, integration, Taylor and Laurent series, residue calculus and conformal mapping, and their applications.\\
 
\noindent {\bf About the Course:} This is an advanced math course which will require
deep understanding of new definitions and concepts as well as mastering new techniques. 
When you learn them, pay particular attention to {\it concrete examples}.
Memorizing definitions without being able to give examples simply won't work.
You will see both computations and proofs in class and in your homework.
This will help you learn to think and express yourselves mathematically and 
to improve your mathematical writing and oral presentation skills.\\

\noindent {\bf Attendance/Class participation:}
I expect that  you attend every class and take notes. Repeated
unexcused absences will lower your participation grade.
Also active participation is encouraged and will contribute
positively to your grade. Such participation includes answering
questions in class, asking good questions in class, presenting
homework solutions, etc. \\

\noindent {\bf Exams:} 
There will be two in-class exams and a comprehensive final exam (see
syllabus for dates). \\

\noindent {\bf Make-Up Policy:} 
If you are unable to take an exam at the scheduled time due to
sickness or an emergency, notify me as soon as possible and we will
work out an appropriate accommodation. If you have a non-emergency
conflict, you must notify me at least a week in
advance in order to have the possibility of making other arrangements.\\

\noindent {\bf Homework:} 
Doing the homework is essential to understanding the material. 
Work on the material outside class. Carefully read the textbook and lecture notes {\it before}
coming to class. While reading use pencil and paper to work on examples and fill in
omitted steps in proofs.  You should seek help when you have trouble.

Solving the assigned problems and writing up the solutions will require a fair amount of time.
I encourage you to work with other students from the class, but each student
must write their solutions independently! 

 Your solutions
should be written in full grammatically correct sentences. 
 When doing calculations, the goal is to explain your
method, not just obtain an answer. When writing a proof, you need to put in enough detail
for the reader to follow your logic easily.
Write legibly. If you are typing, you should use a 12 point
font with 1 inch margins. Staple the pages (in proper order) together.

I will collect homework every Tuesday (starting Sep 4) at the beginning of class. 
No late homework will be accepted!\\


%\newpage

 
\noindent {\bf Blackboard:} We are using Blackboard Learn in this course.
You must log in regularly. There are discussion areas,
grades, handouts, etc. You were automatically signed up when
you registered for this course.\\ To access Blackboard go to
{\tt http://csuohio.edu/elearning/blackboard/bbindex.html} and choose Blackboard Learn
or go directly to
{\tt https://bblearn.csuohio.edu/MACAuth/login.jsp}. Log in using your
CampusPass. See me if you have trouble.\\


\noindent {\bf Grades:} 
Your grade will be determined as follows. 

\medskip


\begin{tabular}{rlp{4cm}ll}
& & & A & 90\% -- 100\%\\
5\% & Class participation & & A$-$ & 85\% -- 89\%\\
45\% & Homework & & B$+$ & 80\% -- 84\%\\
25\% &  Exam I & & B & 77\% -- 79\%\\ 
25\% & Exam II & & B$-$ & 74\% -- 76\%\\
&  & & C$+$ & 70\% -- 73\%\\
------ & ------------------------ & & C & 60\% -- 69\%\\
100\% & TOTAL  & & D & 50\% -- 59\%\\ 
& & & F & below 50\%
\end{tabular}

\bigskip


\noindent {\bf Getting help:}
There are several options for getting help with this course:

$\bullet$ Ask questions about the material or lecture during
class. Chances are that other students are also confused and a brief
question in class may save you hours of misery later.
  
$\bullet$ Ask questions of your peers, either in person or using the
Discussions area in Blackboard. 
%I will also monitor these Discussions area
%and respond to questions posted there.

%$\bullet$ Go to the Math Learning Center (MC 4th floor) for
%tutoring. 
%Details will be available soon.

$\bullet$ Come to my office during office hours. You should be
prepared with specific questions. Try to do as much as possible to
pinpoint your confusion before coming to see me so that we can make
effective use of our time together. Bring any work you have done so
far. If you cannot attend my office hours, feel free to contact me and
make an appointment.

%$\bullet$ Send me an email in Blackboard. Please be as precise as
%possible and include the statement of the problem since I will not
%always the textbook available when I respond. I will do my best to
%respond in a timely manner (within 1-2 days). Do not rely on receiving
%an immediate response. I am not responsible for emails which were
%sent, but did not arrive. If you do not receive a response within a
%couple of days, assume it did not arrive and write again, see me in
%person, or call.
  
$\bullet$ For general academic help the Focus Center (UC 563,
687-5114) runs workshops about time management, test taking skills,
study skills, etc. \\

\noindent{\bf Accommodations for students with disabilities:} CSU provides 
classroom accommodations, auxiliary aids and services to ensure equal educational
opportunities for all students regardless of their disability. For more information contact
the Office of Disability Services at 216-687-2015. Accommodations need to be
requested in advance and will not be granted retroactively.\\

\noindent {\bf Scholastic Dishonesty:} Cheating of any form is not
acceptable and it will be dealt with harshly if detected. If not
detected, you will still be punished since you will be ill-prepared
for future exams in this course or future courses. In addition, it
degrades the value of a CSU degree.
Copying work done by others, in or out of class, is an act of
scholastic dishonesty and it will be prosecuted to the full extent
allowed by university policy. 
CSU policy regarding cheating and plagiarism can be found in the CSU
Code of Student Conduct at
{\tt http://www.csuohio.edu/studentlife/StudentCodeOfConduct2004.pdf}

\newpage
\setlength{\topmargin}{0in}
\setlength{\oddsidemargin}{0in}
\setlength{\textwidth}{6.5in}
\setlength{\textheight}{9.2in}
\setlength{\parindent}{0in}
\pagestyle{empty}

\begin{center}
\begin{bf}
\noindent
{\large Math~416 Complex Analysis -- Fall 2012 -- Weekly Syllabus}\\
\end{bf}
\end{center}

This is a tentative syllabus as of 08/26/12. Lecture topics are subject
to change without notice. Exam days and topics 
are also subject to change, but any changes will be
announced at least a week in advance.\\
  
{\bf Text:} Marsden and Hoffman, {\it
  Basic Complex Analysis}, 3rd edition.\\
(For errata see \verb|http://www.cds.caltech.edu/~marsden/books/Basic_Complex_Analysis.html|) 
 
\begin{tabbing}
{\bf Exam Dates: }
\= Thursday, October 4, in class \hspace{1in} \= Exam 1\\ 
\> Tuesday, November 13, in class  \> Exam 2\\
\> Tuesday, December 11, 4--6 PM \> Final Exam (cumulative)\\
\end{tabbing}


{\bf Week 1: Aug 30, Sep 1}\\
1.1 Introduction to Complex Numbers\\
1.2 Properties of Complex Numbers\\
H Geometry of Complex Functions\\
H Review of the Derivative in Several Real Dimensions\\

{\bf Week 2: Sep 6, 8}\\ 
H Decompositions of Real Linear Maps, Wirtinger Derivatives\\
1.5 Basic Properties of Analytic Functions\\
1.3 Some Elementary Functions\\
1.6 Differentiation of the Elementary Functions\\
\textit{September 9th is the last day to drop}\\

{\bf Week 3: Sep 13, 15}\\ 
H Review of Line Integrals, Double Integrals, and Greens Theorem\\
2.1 Contour Integrals\\
2.2 Cauchy's theorem - A First Look\\

{\bf Week 4: Sep. 20--22}\\ 
2.3 A Closer Look at Cauchy's Theorem\\
2.4 Cauchy's Integral Formula\\

{\bf Week 5: Sep. 27--29}\\
Q\&A for exam\\
EXAM I on Thursday\\ 

{\bf Week 6: Oct. 4--6}\\
2.5 Maximum Modulus Theorem and Harmonic Functions\\
3.1 Convergent Series of Analytic Functions\\
3.2 Power Series and Taylor's Theorem\\

{\bf Week 7: Oct. 4--6}\\ 
3.3 Laurent Series and Classification of Singularities  \\

{\bf Week 8: Oct. 11--13}\\
4.1 Calculation of Residues\\
4.2 Residue Theorem\\
\textit{October 11th is  Columbus Day, no classes }\\

{\bf Week 9: Oct. 18--20}\\
4.3 Evaluation of Definite Integrals\\
Q\&A for exam\\
Exam

{\bf Week 10: Oct. 25--27}\\

{\bf Week 11: Nov. 1--3}\\

{\bf Week 12: Nov. 8--10}\\

{\bf Week 13: Nov. 15--17}\\

{\bf Week 14: Nov. 22--24}\\

{\bf Week 15: Nov. 29--Dec 1}\\

{\bf Week 16: Dec 6--Dec 8}\\


{\bf Week 13: Nov. 19--23}\\
5.1 Basic Theory of Conformal Mappings\\
{\it Thursday is Thanksgiving; no classes}\\

{\bf Week 14: Nov. 26--30}\\ 
%Lab \#2\\ 
5.2 Fractional Linear and Schwarz-Christoffel Transformations \\
5.3 Applications of Conformal Mappings to Laplace's Equation and  Heat  Conduction\\

{\bf Week 15: Dec. 3--Dec. 7}\\ 
5.3 Applications of Conformal Mappings to Electrostatics and Hydrodynamics\\
Review\\   

{\bf Final Exam: Tuesday, Dec 11, 4--6 PM}

\end{document}


